%
% Draft  document wrinkle.tex
% What is known about the genetics of wrinkle score in Merino sheep?
%
 
\documentclass[titlepage]{article}  % Latex2e
\usepackage{graphicx,lscape,subfigure}
\usepackage{bm,longtable}
\usepackage{textcomp}
 

\title{What is known about the genetics of wrinkle score in Merino sheep?}
\author{Neville Jackson and Jim Watts}
\date{12 Oct 2017} 

 
\begin{document} 
 
\maketitle      
\tableofcontents

\clearpage
\section{Introduction} 
There has been a long and often heated debate in the Merino sheep breeding community about the merits of wrinkles or folds in the skin. Starting with Belschner(1937)~\cite{bels:37} reviewing the negative impact of wrinkle on fleece rot and body strike, the debate was taken up by geneticists who set about declaring that wrinkle was a component of clean wool weight and establishing a scoring method with photographic standards (Carter(1943)~\cite{cart:43} , Turner, et al (1953)~\cite{turn:53},  Turner(1956)~\cite{turn:56}, Turner(1958)~\cite{turn:58}). Two selection experiments, for and against wrinkle, were started, one by the NSW Department of Agriclulture, called Folds Plus and Folds Minus lines (Dun(1964)~\cite{dun:64}), and one by CSIRO, called the Wrinkle Plus, and Wrinkle Minus lines (Turner, Brooker, and Dolling(1970)~\cite{turn:70}). The geneticists concluded that there was a possibility that selection for fleece weight would increase wrinkle, and recommended that some culling against wrinkle be conducted in association with selection for fleece weight. There were also negative impilications of folds for fertility (Drinan and Dun (1965)~\cite{drin:65}).

The above paragraph is all about keeping wrinkle at a reasonable level, and controlling its worst side effects with mulesing and chemicals. When the Mules operation  and insecticides became politically incorrect in the late 1990's, a new paradigm suddenly arose.  A number of attempts were made to breed Merino sheep that were completely wrinkle-free, like to original Merinos that came to Australia from England and Spain, before introduction of the  Vermont Merino.  One of these attempts is the SRS group of breeders, who assert that it is not just possible, but essential, that all wrinkle be genetically removed from sheep, befor thay can be successfully selected for the SRS criteria which lead to a different skin type with fine primary fibres, a high S/P ratio due to compound follicles having more branches, fine and long secondary fibres which are more soft or pliable than normal wool. The question arises as to why it is necessary to be wrinkle-free, before breeding towards the SRS type of sheep.

To answer this question, we have to work out how wrinkle development impacts on the type of follicle development process which leads to SRS sheep. We know something about the SRS development process. The pre-papilla cell model of Moore etal(1989)~\cite{moor:89} and Moore etal(1998)~\cite{moor:98} explains why suppression of the size of primary follicles ( and therefore suppression of the primary fibre diameter), leads to vastly greater numbers of secondary derived follicles formed as branches to P and So follicles, leading to compound follicles. What we understand far less, is why or these larger compound follicles grow such a different fibre that is fine, long, low crimped, soft, and lustrous.  We also know something about why these fibres form smaller staples with a different type of staple crimp (Jackson and Watts(2016)~\cite{jack:16}. 

We know next to nothing about the prenatal development of wrinkle. There is one obscure reference in Fraser and Short(1960)~\cite{fras:60}. On page 50, in the section on skin folds it says that body wrinkles appear at around 100 days, first on the dorsal surface and then extending down the sides to the belly. They give an obscure Russian reference (Bogolyubsky(1940)). We should note that 100 days coincides with the period of secondary follicle initiation.


\section{Genetic parameters of wrinkle score}
\subsection{Morley}
Morley(1955) analysed folds scores done at weaning from NSW Department of Agriculture Trangie flocks. The correlations of wrinkle scores with other characters are given in Table~\ref{tab:mor55}
%\documentclass{article}
%\usepackage{lscape}
%\begin{document}

\begin{table}[h]
\centering
\caption{Phenotypic and genetic correlations of wrinkle score with other wool traits from Morley(1955)~\cite{morl:55}}
\label{tab:mor55}
\vspace{0.1in}
\begin{tabular}{|p{1.2in}|p{0.8in}|p{0.8in}|p{0.8in}|}  \hline
  Traits  & Phenotypic correlation  & Genetic correlation  \\  \hline
 Wr x GFW  & 0.31 & 0.42  \\
 Wr x Yield & -0.11 & 0.32 \\
 Wr x CWW & 0.33 & 0.12 \\
 Wr x Bodywt & -0.07 & -0.34 \\
 Wr x StapLen &  -0.18 & -0.47 \\
 Wr x Crimp & 0.04 & 0.21 \\ \hline
\end{tabular}
\end{table}

%\end{document}


\subsection{Brown and Turner}
Brown and Turner(1968)~\cite{brow:68} analysed wrinkle scores from the CSIRO AB1 experiment. The score analysed was a combination of scores for neck and body wrinkle, done after hogget shearing.  Breech wrinkle was not scored as the sheep were mulesed before the scoring took place. The scores were the photographic standards of Turner etal(1953)~\cite{turn:53}.

Wrinkle had a heritability of $0.38 \pm 0.04$ .  The correlations of wrinkle score with all other wool characteristics are givenin Table~\ref{tab:bt68}
%\documentclass{article}
%\usepackage{lscape}
%\begin{document}

\begin{table}[h]
\centering
\caption{Phenotypic and genetic correlations of wrinkle score with other wool traits from Brown and Turner(1968)~\cite{brow:68}}
\label{tab:bt68}
\vspace{0.1in}
\begin{tabular}{|p{1.2in}|p{0.8in}|p{0.8in}|p{0.8in}|}  \hline
  Traits  & Phenotypic correlation  & Genetic correlation  \\  \hline
 Wr x GFW  & 0.27 & 0.18  \\
 Wr x Yield & -0.21 & -0.34 \\
 Wr x CWW & 0.12 & -0.06 \\
 Wr x Bodywt & -0.09 & -0.17 \\
 Wr x Facecovcer & 0.08 &0.14 \\
 Wr x FibreDensity & 0.04 & 0.07 \\
 Wr x FibreDiam & 0.14 & 0.19 \\
 Wr x StapLen &  -0.23 & -0.52 \\
 Wr x Crimp & 0.17 & 0.28 \\ \hline
\end{tabular}
\end{table}

%\end{document}


\subsection{Jackson, Nay, and Turner}
Jackson, Nay and Turner (1975)~\cite{jack:75} analysed wrinkle scores from the CSIRO AB1 experiment in combination with a number of skin and follicle characteristics. The wrinkle scores were as for Brown and Turner(1968)~\cite{brow:68}. The correlations of wrinkle with skin and follicle characteristics are given in Table~\ref{tab:jnt75}.
%\documentclass{article}
%\usepackage{lscape}
%\begin{document}

\begin{table}[h]
\centering
\caption{Phenotypic and genetic correlations of wrinkle score with other wool traits from Jackson, Nay and Turner(1975)~\cite{jack:75}}
\label{tab:jnt75}
\vspace{0.1in}
\begin{tabular}{|p{1.2in}|p{0.8in}|p{0.8in}|p{0.8in}|p{0.8in}|}  \hline
  Traits  & Phenotypic correlation  & Genetic corelation & Environmental correlation  \\  \hline
 Wr x Follicle depth  & 0.06 & 0.04 & 0.07  \\
 Wr x Follicle curvature & 0.36 & 0.68 & 0.16 \\
 Wr x Follicle density & -0.02 & -0.06 & 0.00 \\
 Wr x S/P ratio & 0.22 & 0.29 & 0.18 \\ \hline
\end{tabular}
\end{table}

%\end{document}

These estimates include an enviromental correlation as well as phenotypic and genetic. Environmental correlations should be interpreted as correlations due to all non-genetic effects operating on the two traits.

Note that FibreDensity is a count of fibres per unit area of skin, done on the sheep, not on skin sections. 

\subsection{AB32 Jackson unpublished}
There is a comprehensive genetic analysis of skin characteristics from the CSIRO AB32 follicle selection experiment. These are a different strain of Merino from the AB1 sheep analysed above. It includes wrinkle scores as for Brown and Turner(1968)~\cite{brow:68}, and a wide range of skin and wool characteristice, including diameters of primary and secondary fibres. There were some prelimary estimates in Table 8 of Jackson etal(1990); these should be disregarded as they do not include all the data and were made using outdated estimation procedures. The full analysis of these data, using modern maximum likelihood estimation procedures  and a full relationship matrix, is in an incomplete manuscript (Jackson(2015))~\cite{jack:15}. What we shall do here is extract just those correlations involving wrinkle score. These are given in Table~\ref{tab:ab32}
%\documentclass{article}
%\usepackage{lscape}
%\begin{document}

\begin{table}[h]
\footnotesize
\centering
\caption{Phenotypic , genetic and environmental correlations of wrinkle score with other wool and skin traits from Jackson(2015)~\cite{jack:15}}
\label{tab:ab32}
\vspace{0.1in}
\begin{tabular}{|p{1.2in}|p{0.8in}|p{0.8in}|p{0.8in}|p{0.8in}|}  \hline
  Traits  & Phenotypic correlation  & Genetic correlation & Environmental correlation  \\  \hline
 WrT x Stal  & -0.16 & -0.22 & -0.11  \\
 WrT x Diam  & 0.20 & 0.19 & 0.21 \\
 WrT x Bwt  & 0.03 & -0.34 & 0.28 \\
 WrT x WrN & 0.91 & 0.99 & 0.85 \\ 
 WrT x WrB & 0.91 & 0.99 & 0.85 \\
 WrT x Face & 0.07 & 0.25 & -0.18 \\
 WrT x Gfw & 0.35 & 0.33 & 0.37 \\
 WrT x Yld & -0.27 & -0.36 & -0.20 \\
 WrT x Cww & 0.23 & 0.15 & 0.27 \\
 WrT x Staladj & -0.16 & -0.19 & -0.14 \\
 WrT x Gfwadj & 0.35 & 0.36 & 0.34 \\
 WrT x Cwwadj & 0.22 & 0.18 & 0.25 \\
 WrT x Crimp & 0.20 & 0.46 & -0.23 \\
 WrT x Crwvl & -0.16 & -0.49 & 0.13 \\
 WrT x Crst & 0.08 & 0.44 & -0.27 \\
 WrT x Crstadj & 0.08 & 0.42 & -0.24 \\
 WrT x Crwvt & -0.07 & -0.45 & 0.20 \\
 WrT x Dp & 0.11 & -0.14 & 0.37 \\
 WrT x Ds & 0.21 & 0.47 & -0.00 \\
 WrT x Dps & 0.21 & 0.46 & 0.03 \\
 WrT x DpovDs & -0.03 & -0.32 & 0.53 \\
 WrT x CVDp & 0.14 & -0.04 & 0.27 \\
 WrT x CVDs & 0.04 & -0.32 & 0.26 \\
 WrT x MaxDp & 0.14 & -0.11 & 0.39 \\
 WrT x MinDp & 0.11 & 0.50 & 0.02 \\
 WrT x MaxDs & 0.20 & 0.27 & 0.18 \\
 WrT x MinDs & 0.07 & 0.45 & -0.01  \\
 WrT x SDDp & 0.16 & -0.12 & 0.43 \\
 WrT x SDDs & 0.16 & 0.01 & 0.28 \\
 WrT x SDD & 0.17 & -0.02 & 0.33 \\
 WrT x CVD & 0.05 & -0.33 & 0.30 \\
 WrT x Gt30Dp & 0.09 & -0.10 & 0.28 \\
 WrT x Gt30Ds & 0.17 & 0.19 & 0.16 \\
 WrT x Gt30D & 0.17 & 0.12 & 0.20 \\
 WrT x Fnua & -0.01 & -0.13 & 0.06 \\
 WrT x Fr & 0.16 & 0.27 & 0.10 \\
 WrT x Fnt & -0.00 & -0.33 & 0.19 \\
 WrT x Sarea & 0.03 & -0.39 & 0.30 \\
 WrT x Fd & 0.20 & 0.16 & 0.22 \\
 WrT x Fc & 0.42 & 0.69 & 0.13 \\
 WrT x Fu & 0.40 & 0.68 & 0.21 \\
 WrT x Colour & 0.03 & -0.09 & 0.08 \\
 WrT x Fly & 0.03 & 0.04 & 0.03 \\
 WrT x Flcrot & -0.01 & -0.28 & 0.04  \\
 WrT x Bactst & -0.03 & -0.16 & -0.02 \\
 WrT x MycD & -0.01 & 0.37 & -0.10 \\
 WrT x Bcts & 0.05 & -0.20 & 0.51 \\
 WrT x Bctb & 0.06 & -0.20 & 0.50 \\
 WrT x Weanwt & 0.03 & -0.15 & 0.12 \\
 WrT x NLB & -0.12 & 0.14 & -0.24 \\
 WrT x NLW & -0.12 & 0.02 & -0.19 \\
 WrT x Fnpua & -0.14 & -0.37 & -0.04 \\
 WrT x Fnsua & 0.00 & -0.12 & 0.07 \\
 WrT x Fnpt & -0.13 & -0.54 & 0.07 \\
 WrT x Fnst & 0.01 & -0.32 & 0.19 \\ \hline
\end{tabular}
\end{table}

%\end{document}


The trait symbols in Table~\ref{tab:ab32} are somewhat abbreviated. In Table~\ref{tab:ab32sym1} and Table~\ref{tab:ab32sym2} these are explained in full.
%\documentclass{article}
%\usepackage{lscape,longtable}
%\begin{document}
\begin{center}
\begin{landscape}
\begin{longtable}{p{1.5in}|p{0.8in}|p{1.5in}|p{1.0in}|p{2.5in}}
\caption{Definition of traits measured}  \\
\hline
\label{tab:ab32sym1}
    Trait name & Abbreviation  & Units & Age measured  &  Description \\ 
\hline
\endfirsthead
\multicolumn{5}{c}%
{\tablename\ \thetable\ -- \textit{Continued from previous page}} \\
\hline
    Trait name & Abbreviation  & Units & Age measured  &  Description \\ 
\hline
\endhead
\hline
\multicolumn{5}{r}{\textit{Continued on next page}} \\
\endfoot
\hline
\endlastfoot
%\env{longtable}[p{1.5in}|p{0.8in}|p{1.5in}|p{1.0in}|p{2.5in}]
 Staple length & Stal & mm & 14 months & Length of wool staple 10 months growth \\
 Crimp frequency & Crimp & no per 2.5cm & 14 months & Staple crimp frequency\\
 Fibre diameter & Diam & microns & 14 months & Mean fibre diameter by airflow technique \\
 Greasy Fleece Weight & Gfw & Kg & 14 months & Weight of fleece in shearing shed \\
 Yield & Yld & percentage & 14 months & Percent of clean wool in fleece at 16\% regain \\
 Clean wool weight & Cww & Kg & 14 months & Weight of clean fibre at 16\% regain \\
 Bodyweight & Bwt & Kg & 14 months & Live weight of animal \\
 Neck wrinkle & WrN & score 0-6 (0=plain,6=wrinkled) & 14 months & Score for skin wrinkle on neck region \\
 Body wrinkle & WrB & score 0-5 (0=plain,5=wrinkled) & 14 months & Score for skin wrinkle on body region \\
 Total wrinkle & WrT & sum of WrN and WrB & 14 months & Sum of neck and body wrinkle scores \\
 Face cover & Face & score 1-7 (1=open, 7=muffled) & 14 months & Score for wool cover on the face \\
 Adjusted staple length & Staladj & mm per 365 days & 14 months & Staple length adjusted to a growth period of 365 days \\
 Adjusted clean wool weight & Cwwadj & Kg per 365 days & 14 months & Clean wool weight adjusted to a growth period of 365 days \\
 Adjusted greasy fleece weight & Gfwadj & Kg per 365 days & 14 months & Greasy fleece weight adjusted to a growth period of 365 days \\
 Follicle number per unit area & Fnua & no per $mm_{2}$ & 14 months & No of primary and secondary follicles per $mm_{2}$ from skin biopsy \\
 Follicle $S/P$ ratio & Fr & no units & 14 months & Ratio of no of primary to no of secondary follicles from skin biopsy \\
 Total follicle number & Fnt & no per head x $10^{6}$ & 14 months & No of follicles on the animal (estimated from Fnua and skin surface area) \\
  Surface area & Sarea & $m^{2}$ & 14 months & Smooth skin surface area (estimated from Bwt with no allowance for wrinkle) \\
  Follicle depth & Fd & mm & 14 months & Average follicle depth from skin biopsy and vertical section \\
  Follicle curvature & Fc & score 1-7 (1=straight, 7=curved) & 14 months & Follicle curvature score from skin biopsy and vertical section \\
  Follicle unevenness & Fu & score 1-5 (1=even, 5=uneven) & 14 months & Score for unevenness of follicle depth from skin biopsy and vertical section \\
  Birth weight & Birwt & Kg & day of birth & Weight of lamb on day of birth \\
  Birthcoat score side & Bcts & score 1-6 (1=no halo hairs on side, 6=fully covered) & day of birth & Score for pattern of halo hairs on side of lamb at day of birth \\
  Birthcoat score back & Bctb & score 1-6 (1=no halo hairs on mid backline, 6=dense halo hairs) & day of birth & Score for density of halo hairs on mid backline on day of birth \\
  Weaning weight & Weanwt & Kg & approx 4 months & Weight of lamb on day of weaning \\
  Weaner greasy fleece weight & WeanGfw & Kg & approx 4 months &  Weaner greasy fleece weight at post-weaning shearing \\
  No of lambs born & NLB & no & day of birth & Number of lambs in litter at birth \\
  No of lambs weaned & NLW & no & approx 4 months & Number of lambs in litter at weaning \\
 Greasy wool colour & Colour & score 1-7 (1=white, 7=yellow) & 14 months & Score for greasy yolk colour ignoring any stain present \\
 Flystrike & Fly & score 0-9 (0=absent, 1-9=present to various degrees) & 14 months & Score for presence or absene of flystrike at any site \\
 Fleece rot & Flcrot & score 0-9 (0=absent, 1-9=present to various degrees) & 14 months & Score for presence or absence of fleece rot \\
 Bacterial stain & Bactst & score 0-9 (0=absent, 1-9=present to various degrees) & 14 months & Score for presence or absence of bacterial stain \\
 Mycotic dermatitis & MycD & score 0-9 (0=absent, 1-9=present to various degrees) & 14 months & Score for presence or absence of mycotic dermatitis \\
  Mean diameter of primaries & Dp & microns & 14 months & Mean diameter of primary fibres from biopsy and horizontal section \\
  Mean diameter of secondaries & Ds & microns & 14 months & Mean diameter of secondary fibres from biopsy and horizontal section \\
  Mean diameter of primaries and secondaries & Dps & microns & 14 months & Mean diameter of primary and secondary fibres from biopsy and horizontal section \\
  Primary to secondary diameter ratio & DpovDs & no units & 14 months & Ratio of mean diameter of primary fibres to mean diameter of secondary fibres \\
  CV of primary diameter & CVDp & no units & 14 months & Coefficient of variation of primary fibre diameter \\
  CV of secondary diameter & CVDs & no units & 14 months & Coefficient of variation of secondary fibre diameter \\
  Maximum diameter of primaries & MaxDp & microns & 14 months & Diameter of the largest primary fibre \\
  Minimum diameter of primaries & MinDp & microns & 14 months & Diameter of the smallest primary fibre \\
  Maximum diameter of secondaries & MaxDs & microns & 14 months & Diameter of the largest secondary fibre \\
  Minimum diameter of secondaries & MinDs & microns & 14 months & Diameter of the smallest secondary fibre \\
  SD of primaries & SDDp & microns & 14 months & Standard deviation of primary fibre diameter \\
  SD of secondaries & SDDs & microns & 14 months & Standard deviation of secondary fibre diameter \\
  SD of all fibres & SDD & microns & 14 months & Standard deviation of primary and secondary fibre diameter \\
  CV of all fibres & CVD & no units & 14 months & Coefficient of variation of primary and secondary fibre diameter \\
  Primaries greater than 30 microns & Gt30Dp & frequency & 14 months & Proportion of primary fibres exceeding 30 microns in diameter \\
  Secondaries greater than 30 microns & Gt30Ds & frequency & 14 months & Proportion of secondary fibres exceeding 30 microns in diameter \\
  Fibres greater than 30 microns & Gt30D & frequency & 14 months & Proportion of fibres exceeding 30 microns in diameter \\

\end{longtable}
\end{landscape}
\end{center}
%\end{document}

%\documentclass{article}
%\usepackage{lscape,longtable}
%\begin{document}
\begin{center}
%\begin{landscape}
\begin{table}[p]
\caption{Definition of traits calculated from measured traits using a known functional relationship}  
\label{tab:ab32sym2}
\vspace{0.1in}
\begin{tabular}{p{1.5in}|p{0.8in}|p{1.2in}|p{2.0in}} \hline
    Trait name & Abbreviation  & Units &  Functional relationship \\ 
\hline
 & & & \\
Primary follicle density & Fnpua & no per $mm^{2}$ & $Fnpua = \frac{Fnua}{(Fr + 1)}$ \\
 & & & \\
Secondary follicle density & Fnsua & no per $mm^{2}$ & $Fnsua = \frac{(Fr)(Fnua)}{(Fr + 1)}$ \\
 & & & \\
Total primary follicle number & Fnpt & No per head x $10^{6}$ & $ Fnpt = (Fnpua)(Sarea)$ \\
 & & & \\
Total secondary follicle number & Fnst & No per head x $10^{6}$ & $ Fnst = (Fnsua)(Sarea)$ \\
 & & & \\
Crimp wavelength & Crwvl & mm & $ Crwvl = \frac{25.4}{Crimp}$ \\
 & & & \\
Crimps per staple & Crst & number & $Crst = Crimp * Stal / 25.4 $ \\
 & & & \\
Crimps per 365 days (crimp frequency in time) & Crstadj & number per 365 days & $Crstadj = Crimp * Staladj / 25.4 $ \\
 & & & \\
Crimp wavelength in time & Crwvt & days & $Crwvt = \frac{365}{Crstadj} $ \\
 & & & \\
\hline

\end{tabular}
\end{table}
%\end{landscape}
\end{center}
%\end{document}

I have deliberately chosen not to present standard errors. These were between 0.03 and 0.06 for genetic and environmental correlations, and between 0.01 and 0.03 for pphenotypic correlations. In other words any correlation bigger than about 0.10 ( or less than -0.10) would be significantly different from zero. The heritability of wrinkle score in these analyses was $0.45 \pm 0.01$, so there is plenty of genetic variation behind these correlations.

There is a general agreement of these correlations with previous estimates. Theere is no reason not to take these as the best available estimates.

\subsection{Summary of correlations}
There seem to be a number of groups of relationships
\begin{itemize}
\item wrinkle is not strongly genetically correlated with Dp or birthcoat, but there is a positive environmental correlation. This suggests wrinkle may not interfere with primary follicle initiation
\item wrinkle is positively genetically correlated with Ds and S/P ratio. This is the relation which suggesta wrinkle may affect prepilla cell dynamics. The correlation with overall diameter was large and positive for diameter measured on skin sections, but small and positive for diameter measured on wool. 
\item wrinkle is very strongly correlated with follicle curvature, follicle unevenness, crimp frequency , and staple length. This groups of relations suggests wrinkle affects bulb cell development into the cortex strructure
\item wrinkle is genetically correlated with GFW and Yield, but not strongly with CWW. It would seem that wrinkled sheep produce more wax, but not a lot more wool, if any. 
\item wrinkle is negatively associated with bodyweight and surface area. The surface area calculated in these data accounts for body size but not wrinkle. It is therefore suspect, and so are the total follicle counts (Fnt, Fnpt, Fnst)
\end{itemize}


\section{Responses to selection for and against wrinkle}
We can check on genetic correlations by seeing what happens to other traits if we select for  or against wrinkle.
\subsection{CSIRO wrinkle plus and minus groups}
These lines are reported in Turner, Brooker, and Dolling(1970)~\cite{turn:70}.
The selectioe led to a difference in GFW, Yield, Bodyweight, Face score, Staple Length and Crimp Frequency. There was little or no difference in CWW, Fibre Number, or Fibre diameter.  There was of course a difference in wrinkle.

\subsection{NSWDA folds plus and folds minus groups}
These groups were selected for and against wrinkle scored at weaning. There are reported correlated responses in reproduction, but we could only locate the following short account of wool differences in Richards, Jessica, etal (2009)~\cite{rich:09}.
\begin{verbatim}
Table 1. Correlated responses with wrinkle (data from Folds Plus and
       Folds Minus flocks at Trangie, 1951 to 1971, c/w control line)

    Age        Trait                        Selection lines

                                       Folds Plus           Folds Minus

    Hogget    Greasy fleece wt         + 6 %                - 13 %

              Clean fleece wt          - 1 %                - 10 %

              Fibre diameter           + 0.1 um             + 0.3 um

              Body weight              - 1 %                + 3 %

 

     Adult    Lambs weaned             - 26 %               - 2 %
              per ewe joined


The authors also state that if a 1 wrinkle score reduction was
 introduced into an index placing equal emphasis on reducing
 fibre dimeter and increasing fleece weight, 0.8 skin wrinkle
 reduction in 10 years would be achieved but with a 70 % 
reduction in the index trait gains .
\end{verbatim}

\section{Search for nonadditive genetic variance}
The CSIRO AB1 dataset was used to examine whether for wrinkle score there was evidence of any genetic variance other than individual additive genetic. There was a suggestion of some maternal additive genetic variance ( about 10 percent) and also some individual sexlinked additive genetic variance ( again about 10 percent). So most of the variation in wrinkle was environmental or individual additive genetic. 

What this means is that the genetic correlations reported above, which are all for individual additive genetic (co)variances, are likely to cover all the important large genetic relationships. 

\section{How might wrinkle development affect the dynamics of pre-papilla cell growth?}
Given that the effect of wrinkle on follicle development seems to be at the secondary follicle initiation stage, the following mechanisms are possible
\begin{itemize}
\item wrinkles make space for more So sites to form - this contradicts what Philip Moore has demonstrated - that sites for P and So follicles are definite positions, not just empty space.
\item wrinkles lead to more pre-papilla cells being used at each So site, leading to larger So follicles and fibres and fewer leftover cells.
\item wrinkles lead to earlier So initiation so the pre-papilla cell bank is raided earlier leaving fewer cells left to form Sd follicles.
\item wrinkles lead to reduced or early terminated pre-papilla cell division
\item wrinkles and sites are genetically correlated
\end{itemize}

We really have no idea whether any of the above mechanisms operate

\section{Discussion}
As summarized above there would seem to be several groups of relationships involving wrinkle. Only the correlations with Ds and S/P suggest involvement of wrinkle in the pre-papilla cell dynamics. 

The correlation with follicle curvature and crimp suggest something else. Follicle curvture is the same thing as intrinsic fibre curvature, which in turn depends on the structure of the fibre cortex. Cortex structure is determined by what happens to bulb cells as they differentiate and form the fibre. If the fibres are more curved, staple length is shorter, but fibre length is not necessarily less. So this group of correlations probably has nothing to do with pre-papilla cell dynamics, unles, of cource, the cells in the papilla somehow direct how bulb cells differentiate.

Wrinkled sheep make more wax in the fleece. So they have more sebacious glands. That relatesd to higher S/P ratio, and probably to larger compound follicles. Because compound follicles are the endpoint of pre-papilla-cell dynamics, the  lower Yield of wrinkled sheep may also indicate that wrinkle interferes with pre-papilla cell development.

\begin{thebibliography}{99}

\bibitem{bels:37}
Belschner H.G. (1937) "A review of the sheep blowfly problem in New South Wales and observations on fleece rot and body strike in sheep, particularly in regards to their incidence, type of sheep susceptible, and economic importance". NSW Department of Agriculture, Science Bulletin No 54, 1937, pp 1-95

\bibitem{brow:68}
Brown, G.H., and Turner, Helen Newton. (1968) Response to selection in Australian Merino sheep. II. Estimates of phenotypic and genetic parameters for some production traits in Merino ewes and an analysis of the possible effects of selection on them. Aust. J. Agric. Res. 19:303-22

\bibitem{cart:43}
Carter, H.B. (1943) Studies in the biology of the skin and fleece of sheep. CSIRO (Aust) Bull. No. 164
 
\bibitem{dun:64}
Dun, R.B. (1964) Skin folds and Merino breeding. 1. The net reproductive rates of flocks se;lected for and against skin fold. Aust J. Exp. Agric. Anim. Husb. 4:376-385

\bibitem{drin:65}
Drinan, J.P. and Dun, R.B. (1965) Skins folds and Merino breeding. 3. The association between skin fold and productivity of Merino ewes in fourteen NSW flocks. Aust.J.Exp.Agric.Anim.Husb. 5(19): 345-352

\bibitem{fras:60}
Fraser A.S and Short B.F. (1960) The Biology of the Fleece. Animal Research Laboratories Technical Paper No 3. CSIRO Melbourne 1960.

\bibitem{jack:75}
Jackson, N., Nay, T, and Turner, Helen Newton (1975) Response to selection in Australian Merino sheep. VII Phenotypic and genetic parameters for some wool follicle characteristics and their correlation with wool and body traits. Aust. J. Agric. Res. 26:937-57

\bibitem{jack:15}
Jackson, N. (2015) Genetic relationship betweeen skin and wool traits in Merino sheep. Incomplete manuscript.

\bibitem{jack:16}
Jackson, N. and Watts, J.E. (2016) Staple crimp formation in the fleece of Merino sheep. Unpublished manuscript, 18 May 2016.

\bibitem{moor:89}
Moore G.P.M., Jackson, N., and Lax, J. (1989) Evidence of a unique developmental mechanism specifying bot wool follicle density and fibre size in sheep selected for single skin and fleece characters. Genet. Res. Camb. 53:57-62

\bibitem{moor:98}
Moore, G.P.M., Jackson, N., Isaacs, K., and Brown, G (1998) J. Theoretical Biology 191:87-94

\bibitem{morl:55}
Morley F.H.W (1955) Selection for economic characters in Merino sheep. V. Further estimates of phenotypic and genetic parameters. Aust. J. Agric. Res. 6:77-90

\bibitem{nayj:73}
Nay, T. and Jackson, N. (1973) Effect of changes in nutritional level on the depth and curvature of wool follicles in Australian Merino sheep. Aust. J. Agric. Res. 24:439-447

\bibitem{onio:62}
Onions, W.J. (1962) Wool: an introduction to its properties, varieties, uses
     and production. Ernest Benn limited, London, 1962

\bibitem{rich:09}
Richards, Jessica, etal (2009) Breeding Merinos for less breech strike. June 2009. Primefact 918. NSWDPI.

\bibitem{rprog:13}
R Core Team (2013). R: A language and environment for statistical
  computing. R Foundation for Statistical Computing, Vienna, Austria.
  ISBN 3-900051-07-0, URL http://www.R-project.org/.

\bibitem{turn:53}
Turner, Helen Newton, Hayman, R.H., Riches, J.H., Roberts, N.F., and Wilson, L.T. (1953) Physical definition of sheep and their fleece for breeding and husbandry studies: with particular reference to Merino sheep. CSIRO Div. Anim. Hlth. Prod. Div. Rept. No. 4 (Ser SW-2 mimeo)

\bibitem{turn:56}
Turner, Helen Newton (1956) Anim Breed Abstr. 24:87-118

\bibitem{turn:58}
Turner, Helen Newton (1958) Aust J. Agric. Res. 9:521-552

\bibitem{turn:70}
Turner, Helen Newton, Brooker M.G. and Dolling, C.H.S (1970) Response to selection in Australian Merino sheep. III Single character selection for high and low values of wool weight and its components. Aust.J.Agric.Res. 21:955-84
\end{thebibliography}
\end{document}
