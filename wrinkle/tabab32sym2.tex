%\documentclass{article}
%\usepackage{lscape,longtable}
%\begin{document}
\begin{center}
%\begin{landscape}
\begin{table}[p]
\caption{Definition of traits calculated from measured traits using a known functional relationship}  
\label{tab:ab32sym2}
\vspace{0.1in}
\begin{tabular}{p{1.5in}|p{0.8in}|p{1.2in}|p{2.0in}} \hline
    Trait name & Abbreviation  & Units &  Functional relationship \\ 
\hline
 & & & \\
Primary follicle density & Fnpua & no per $mm^{2}$ & $Fnpua = \frac{Fnua}{(Fr + 1)}$ \\
 & & & \\
Secondary follicle density & Fnsua & no per $mm^{2}$ & $Fnsua = \frac{(Fr)(Fnua)}{(Fr + 1)}$ \\
 & & & \\
Total primary follicle number & Fnpt & No per head x $10^{6}$ & $ Fnpt = (Fnpua)(Sarea)$ \\
 & & & \\
Total secondary follicle number & Fnst & No per head x $10^{6}$ & $ Fnst = (Fnsua)(Sarea)$ \\
 & & & \\
Crimp wavelength & Crwvl & mm & $ Crwvl = \frac{25.4}{Crimp}$ \\
 & & & \\
Crimps per staple & Crst & number & $Crst = Crimp * Stal / 25.4 $ \\
 & & & \\
Crimps per 365 days (crimp frequency in time) & Crstadj & number per 365 days & $Crstadj = Crimp * Staladj / 25.4 $ \\
 & & & \\
Crimp wavelength in time & Crwvt & days & $Crwvt = \frac{365}{Crstadj} $ \\
 & & & \\
\hline

\end{tabular}
\end{table}
%\end{landscape}
\end{center}
%\end{document}
